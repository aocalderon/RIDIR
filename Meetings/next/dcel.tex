\documentclass{beamer}

\usefonttheme{professionalfonts} % using non standard fonts for beamer
\usefonttheme{serif} % default family is serif

%\usepackage{hyperref}

%\usepackage{minted}

\usepackage{animate}

\usepackage{graphicx}

\def\Put(#1,#2)#3{\leavevmode\makebox(0,0){\put(#1,#2){#3}}}

\usepackage{color}

\usepackage{tikz}

\usepackage{amssymb}

\usepackage{enumerate}


\newcommand\blfootnote[1]{%

  \begingroup

  \renewcommand\thefootnote{}\footnote{#1}%

  \addtocounter{footnote}{-1}%

  \endgroup

}

\makeatletter

%%%%%%%%%%%%%%%%%%%%%%%%%%%%%% Textclass specific LaTeX commands.

 % this default might be overridden by plain title style

 \newcommand\makebeamertitle{\frame{\maketitle}}%

 % (ERT) argument for the TOC

 \AtBeginDocument{%

   \let\origtableofcontents=\tableofcontents

   \def\tableofcontents{\@ifnextchar[{\origtableofcontents}{\gobbletableofcontents}}

   \def\gobbletableofcontents#1{\origtableofcontents}

 }

%%%%%%%%%%%%%%%%%%%%%%%%%%%%%% User specified LaTeX commands.

\usetheme{Malmoe}

% or ...

\useoutertheme{infolines}

\addtobeamertemplate{headline}{}{\vskip2pt}



\setbeamercovered{transparent}

% or whatever (possibly just delete it)

\makeatother

\begin{document}
\title[SDCEL report]{A Scalable DCEL implementation}
\author[AC]{Andres Calderon}
\institute[Spring'20]{University of California, Riverside}
\makebeamertitle
\newif\iflattersubsect

\AtBeginSection[] {
    \begin{frame}<beamer>
    \frametitle{Outline} 
    \tableofcontents[currentsection]  
    \end{frame}
    \lattersubsectfalse
}

\AtBeginSubsection[] {
    \begin{frame}<beamer>
    \frametitle{Outline} 
    \tableofcontents[currentsubsection]  
    \end{frame}
}

\begin{frame}{Improvements on CGAL implementation...}
    \begin{itemize}
        \item Discard the use of \textit{insert\_non\_intersecting\_curve}.  It does not create the DCEL correctly...
        \item Given a try to polylines insertion.  It inserts a sequence of segment lines and use the previous segment to locate the position of the next one.
    \end{itemize}
\end{frame}

\begin{frame}{Performance...}
    \begin{itemize}
        \item The reference executes 97.6 s for the construction of only one arrangement of uniform random segments with 1,366,364 edges.
        \item New implementation using polylines in the CA polygon dataset (avg of 5 runs): \\
        \begin{tabular}{lll}
            \hline
             & \textbf{Number of edges} & \textbf{Time(s)} \\
            \hline
            CA 2000 & 1,002,370  & 30.61\\
            CA 2010 & 2,896,123  & 659.36\\
            \hline \\
        \end{tabular} 
      \end{itemize}
\end{frame}

\end{document}
