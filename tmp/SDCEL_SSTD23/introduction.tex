\section{Introduction}
The use of spatial data structures is ubiquitous in many spatial applications, ranging from spatial databases to computational geometry, robotics and geographic information systems \cite{samet-book}. 
%It has important advantages to offer to current spatial algorithms in particular the application of spatial indexes and opportunities to exploit proximity among the data.  
Spatial data structures have been used to improve the efficiency of various spatial queries, such as spatial joins, nearest neighbors, voronoi diagrams and robot motion planning.
Examples include grids \cite{gridfile}, R-trees \cite{rtree, rstar}, quadtrees \cite{quadtree}, etc.
%are widely reported in literature, other structures have been less the focus of attention.  In particular, 
There are also \textit{edge-list}  structures that have been typically utilized in applications as topological computations in computational geometry \cite{berg_computational_2008}.
%but their employment have been limited to very specific domains.  The use of such kind of structures has been reported in applications 
%like obtaining silhouettes of polyhedra, efficient Minkowksi sums, offset of polygons and diverse types of triangulations \cite{berg_computational_2008}.
%mainly in computational geometry projects.

The most commonly used data structure in the edge-list family is the Doubly Connected Edge List (DCEL).  A DCEL \cite{muller_finding_1978, preparata_computational_1985} is a data structure which collects topological information for the edges, vertices and faces contained by a surface in the plane. The DCEL and its components represent a planar subdivision of that surface. In a DCEL, the faces (polygons) represent the cells of the subdivision; the edges are boundaries which divide adjacent faces; and the vertices are the point endings between adjacent edges.
In addition to geometric and topological information a DCEL can be enhanced to provide further information.  For instance, a DCEL storing a thematic map for vegetation can store also information about the type and height the trees around the area \cite{berg_computational_2008}. 

The DCEL data structure has been used in various applications. For instance, the use of connected edge lists is cardinal to support polygon triangulations and their applications in surveillance (the Art Gallery Problem \cite{chvatal_combinatorial_1975, orourke_art_1987}) and robot motion planning (Minkowski sums \cite{berg_computational_2008, chew_convex_1993}).  DCELs are also used to perform polygon unions (for example, on printed circuit boards to support the simplification of connected components in an efficient manner \cite{fogel_cgal_2012}) as well as the computation of silhouettes from polyhedra \cite{fogel_cgal_2012, berberich_arrangements_2010} (applied frequently in computer vision and 3D graphics modelling \cite{boguslawski_modelling_2011}).

Edge-list data structures have also been utilized for the creation of thematic \textit{overlay maps}. In this problem, the input contains the DCELs of two polygon layers each capturing geospatial information and attribute data for different phenomena and the output is the DCEL of an overlay structure that combines the two layers into one. In many application areas such as ecology, economics and climate change, it is important to be able of join the input layers and match their attributes in order to unveil patterns or anomalies in data which can be highly impacted by location. Several operations can then be easily computed given an overlay; for instance, the user may want to find the \textit{intersection} between the input layers, identify their \textit{difference} (or symmetric difference), or create their \textit{union}. 

Spatial databases have been using spatial indexes (R-tree \cite{rtree, rstar}) to store and query polygons. Such methods use the \textit{filter and refine} approach where a complex polygon is abstracted by its Minimum Bounding Rectangle (MBR) that is inserted in the R-tree index. Finding the intersection between two polygon layers each indexed by a separate R-tree is then reduced to finding the pairs of MBRs from the two indexes that intersect (filter part). This is followed by the refine part, which, given two MBRs that intersect needs to compute the actual intersections between all the polygons these two MBRs contain. While MBR intersection is simple, computing the intersection between a pair of complex real-life polygons is a rather expensive operation (a typical 2020 US census track is a polygon with hundreds of edges). 

Moreover, using DCELs for overlay operations offers the additional advantage that the result is also a DCEL which can then be directly used for subsequent operations. For example, one may want to create an overlay between the intersection of two layers with another layer and so on. 

%\textcolor{red}{clarify paragraph} Furthermore, most of distributed techniques that have been used in this matter are oriented to a specific spatial operation (intersection, union or difference) and they have to be run from scratch if other operation is required.  In addition, current parallel techniques divide the data into partitions and replicate features if needed in order to solve the problem locally.  It could potentially increase the size of the problem.  

Even though the DCEL has important advantages for implementing overlay operations, current approaches are sequential in nature. This is problematic considering layers with thousands of polygons. For example, layers representing the 2020 US census tracks contains around 72K polygons; the execution time for computing the overlay over such files becomes prohibitive, taking days to get the results on a stock laptop. To the best of our knowledge there is no scalable solution to compute overlays over DCEL layers.

%Even sequential DCEL implementation are not new, nowadays, with the scale and volume of available geodata, the rise of big (spatial) data makes necessary to count with fast and efficient techniques for spatial analysis.  For example, today GIS researchers have to deal with spatial operations between layers collecting thousand of counties boundaries at nation-wide level.  The versatility and efficiency of the spatial methods is cardinal for their studies. Given the advantages shown by the DCEL data structure, it should be interesting to count with intermediate data structures that not just exploit the advantages of distributed frameworks but also allow multiple map overlay queries at the same time. 

%In addition, we already know that current topological data structures are common in computational geometry.  However, most implementations are sequential and they do not scale appropriately on large spatial datasets.  Moreover, distribute alternatives like parallel spatial joins add complexity due to spatial partitioning and replication and they are unable to run more than one operation once they have been created.

In this paper we describe the design and implementation of a scalable and distributed approach to compute the overlay between two DCEL layers. We first present a partition strategy that guarantees that each partition collects the required data from each layer DCEL to work independently, thus minimizing duplication and transmission costs over 2D polygons.  In addition, we present a merging procedure that collects all partition results and consolidates them in the final combined DCEL.  
%In section \ref{sec:strategy} we present and discuss the implementation of a novel strategy to divide the input data and build a distribute and scalable DCEL to be used 
Our approach has been implemented in a parallel framework (i.e., Apache Spark). 

Implementing a distributed overlay DCEL creates novel problems. First, there are potential challenges which are not present in the sequential DCEL execution. For example, the implementation should consider features such as \textit{holes} and \textit{multi-polygons} which could lay on different partitions.  Such features need to be connected with their components residing in other partitions so as to not compromise the correctness of the combined DCEL.  %Appropriate mechanism should be applied to solve these possible cases.

Secondly, once a distributed overlay DCEL has been built, it must support a set of binary overlay operations (namely \textit{union, intersection, difference} and \textit{symmetric difference}) in a transparent manner.  That is, such operators should take advantage of the scalability of the overlay DCEL and be able to run also in a parallel fashion.  Additionally, users should be able to apply the various operators multiple times without the need of rebuild the DCEL data structure.  %That means, that the binary operators should be applied on top of the distributed DCEL in a parallel fashion and , once it is done, the local results should be integrated together in order to remove any duplication and merging partial results obtained on contiguous partitions.

The rest of this paper is organized as follows. Section \ref{sec:related} presents related work while Section \ref{sec:prelim} discusses the basics of DCEL and the sequential algorithm.
In Section \ref{sec:methods} we present a partitioning scheme that enables parallel implementation of the overlay computation among DCEL layers; we also discuss the challenges presented in the DCEL computations by distributing the data and how to solve them efficiently. 
Two important optimizations are introduced in Section \ref{sec:alternative_methods}.
An extensive experimental evaluation appears in Section \ref{sec:experiments}, while Section \ref{sec:conclusions} concludes the paper.
%could arise.  We integrate a solution to solve cases such as isolated holes or empty partitions which could lost reference data during the distribution. We present an algorithm to detect and track partitions with orphans components and systematically search their neighborhood for related features.  

%Section \ref{sec:overlay} provides the details of the implementation of the binary overlay operators which query the DCEL.  The operations for intersection ($A \cap B$), union ($A \cup B$),  difference ($A - B$) and symmetric difference ($A \triangle B$) are available.  They are executed locally at each partition of the DCEL and then collecting and merging the results to provide the final answer.  The parallel execution take advantage of the distribution and efficiently exploits the locality to present the results.

%At the best of our knowledge, there is not a distributed and scalable DCEL implementation available that meet those criteria.  We think it could be a great tool to support key challenges and operations in Geoscience today.