\section{Related Work}
\label{sec:related}
The fundamentals of the DCEL are explained in the seminal paper \cite{muller_finding_1978} and \cite{preparata_computational_1985}.  Authors highlight among the main advantages of DCELs the opportunity of capturing topological information and allowing multiple overlay operations once the DCEL is created.  

An important reference for DCEL description and design should be \cite{berg_computational_2008}.  It states a DCEL can be constructed in $\mathcal{O}(n log(n))$ time using $\mathcal{O}(n)$ additional memory where $n$ is the number of vertices in the input layer \cite{freiseisen_colored_1998}. Once created, the DCEL allows binary overlay operations in $\mathcal{O}(n)$ time using $\mathcal{O}(n)$ additional space. 

Nowadays, sequential implementations of DCEL have been presented and used in diverse applications \cite{barequet_dcel_1998, boltcheva_topological-based_2020, freiseisen_colored_1998}. Although there is not reference to distributed DCEL implementations, other dynamic parallel data structures has been described \cite{challa_dd-rtree_2016, sabek_spatial_2017, li_scalable_2019}.  However, spatial indexes and spatial joins could support overlay operations in certain way but just for an individual operator at a time. 

Another works present parallel map overlay algorithms whose focus on GPU and multi-core architecture \cite{franklin_data_2018, magalhaes_fast_2015, puri_efficient_2013, puri_mapreduce_2013}.  Most of them use scan lines to partition the set of edges and hierarchical tree structures (octree or rtree) to partition the input data. However, as previous works, they test the implementation only with the intersection overlay operation. Even though, these works focus on solutions aim to multi-core architectures or GPGPU models which scalability can be affected by very large datasets.  In addition, implementations over the traditional multi-core ecosystem could not take total advantage of modern distributed memory frameworks such as Apache Spark.

Currently, few sequential implementations are available in the market.  Most important are LEDA\footnote{\url{https://www.algorithmic-solutions.com/}} \cite{mehlhorn_leda_1995}, Holmes3D\footnote{\url{http://www.holmes3d.net/graphics/}} \cite{holmes_dcel_2021} and CGAL\footnote{\url{https://www.cgal.org/}} \cite{fogel_cgal_2012}.  LEDA and Holmes3D are close-source software and limited access.  On the other hand, CGAL is an open-source project with a large trajectory in the area of computational geometry offering a wide-ranging number of packages and modules to support diverse areas offering a solid support for DCEL construction.
